% !TeX spellcheck = en_US
% !TeX encoding = UTF-8
% !TeX root = ../thesis.tex

\chapter{Related Work}

\section{Policy Search Algorithms}
Generally model-free policy search algorithms have have been used
for their ease of use, at often being black box optimizers.

For model-free policy search the agent generates trajectories
evaluates them and uses the reward and the trajectories to
iteratively update the policy such that the expected reward is maximized.

There are different methods for computing the update of the
policy, common strategies are based on gradient ascent
\citet{peters2006policy}.

- natural gradient by NES (update direction of parameters like standard
gradient but still in bound by KL-divergence)

- evolutionary strategies

- expectation-maximization algorithms

The MORE algorithm is part of information theoretic approaches
to PS.
Information theoretic approaches use
the Kullback-Leibler Divergence in an optimization problem
to bound the distance of the old policy to the new one.
There are different techniques to solve the Integrals that arise
when optimizing the problem.
The REPS algorithm proposed in \cite{peters2010relative} uses a
sample-based approximation for the KL-divergence.
Using Taylor-expansions for the KL-divergence resulted in the natural
evolutionary strategies (NES) \cite{wierstra2014natural}.

Some additions to the original MORE have been explored like formulating
a contextual version CMORE \citet{tangkaratt2017policy}, enabling learning from high-dimensional
context variables like camera images.

\section{Surrogate modeling}
Model-based methods for approximating the objective function
with a local surrogate have been used in
derivative free optimization \citet{nocedal2006numerical}.
Local surrogate models have been used in trust region methods like \citet{powell2009bobyqa}. These methods maintain a point estimate
and a trust region around this point instead
of a search distribution, they update the point estimate by optimizing
the surrogate and staying in the trust region.

Recursive estimation with methods like
the Kalman Filter is extensively used in robotics, mostly to deal with
noisy measurements. Probabilistic methods from
a Bayesian especially with a focus on decision making
is introduced in \citet{thrun2002probabilistic}.

