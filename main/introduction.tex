% !TeX spellcheck = en_US
% !TeX encoding = UTF-8
% !TeX root = ../thesis.tex

\chapter{Introduction}

\section{Motivation}
Robots are already used extensively in industry to form production chains,
where they perform the same task over and over.
These robots are being programmed and fine tuned
by a human engineer which requires experience and expertise. \\
Recently there has been a new development of robots
becoming part of our daily life's, for example in the form of 
vacuum cleaner and lawn mowers . In the future areas like care giving and
everyday assistance and household robots may
become more established \citet{schaal2007new}.
This poses a dramatic shift from the prior uses of robots in industry, where
they mainly worked in isolated and predefined contexts.
Especially countries like Japan, facing the problem of an
aging population, put increasing effort to make robots
viable in these domains.  \\
In daily life the robots are confronted with different challenges,
like  adapting to different like lightening conditions and
objectives being moved around.
The robots which are currently already on the market
like vacuum cleaners and lawn mowers have
either a ``one size fits all'' approach or need a special setup
in software or hardware. To solve this issue
machine learning and especially
reinforcement learning will be key technologies to enable robots
to adjust to dynamic and stochastic environments.

- TODO: focus on motor skills (movement primitives)

Robotics and reinforcement learning complement each other
with robotics providing a great, and reinforcement learning providing
the framework for formulating solutions, the 
relationship may be similar to the one of  math and physics
\citet{kober2013reinforcement}. \\
Compared to other domains like board games Go, or video games Atari
Robotics  poses special challenges for reinforcement algorithms. Making
it necessary to explore methods specialized to the inherent
challenges of Robotics, which include:

\begin{itemize}
\item high dimensional state and action space
\item obtaining real world samples
\item goal specification
\end{itemize}

In general Robots have potential to transform our society, by
bringing robots from the factories into our homes there will be many
possibilities for improvement and challenges likewise.

- TODO: include pictures of robots solving tasks with RL


\section{Contribution}
Policy search algorithms have shown promise as an alternative
to value function-based reinforcement learning, especially for learning motor
skills in robotics \citet{deisenroth2013survey}.
The MORE algorithm as a policy search method based on
information theoretic updates is introduced in \citet{abdolmaleki2015model}.
The key idea of MORE is to learn a surrogate Model of
the objective function to efficiently computing the updates to
the policy in closed form. One of
the contributions of this thesis is to explore recursive estimation 
for learning the surrogate model of the objective function to increase
sample efficiency. Besides that we aim to improve the  runtime of
the algorithm with this approach.
We focus mainly on classical methods of parameter estimation and filtering
like the Recursive Least Squares and the Kalman Filter,
considering more advanced methods is part of future work.
We benchmark the our version of the MORE algorithm on test functions and
simple planar reaching tasks and compare
them to previous results.

\section{Structure of Thesis}
The remainder of this thesis is structured as follows:

\textbf{Chapter 2:} In the fundamentals chapter we lay the foundation for
the thesis, first introducing the basics of Reinforcement Learning and
then focusing on the specifics of applying RL to robotic tasks.
Then We focus on policy search as one method for
solving the robot learning problem.
Next we introduce the original MORE algorithm and finally
we look at Bayesian filtering.

\textbf{Chapter 3:} Review of the related work in the field.

\textbf{Chapter 4:} In this chapter we motivate the idea of
using recursive estimation for learning the surrogate model
and then review methods used for data preprocessing.
Finally we develop our approach of  connecting
the MORE algorithm with recursive parameter estimation. 

\textbf{Chapter 5:} In the evaluation chapter we conduct experiments
with the algorithms on several tests function and some simple reaching tasks
for a planar robot. We compare our algorithm with original MORE and
other stochastic search algorithms.

\textbf{Chapter 6:} We conclude the thesis with a summary of
the achieved results and an outlook on future work.