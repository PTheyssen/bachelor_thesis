% !TeX spellcheck = en_US
% !TeX encoding = UTF-8
% !TeX root = ../thesis.tex

\chapter{Introduction}

\section{Motivation}
Robots are already used extensively in industry to form production chains,
where they perform the same task over and over.
These robots are being programmed and fine tuned
by a human engineer which requires experience and expertise. \\
Recently, there has been a new development of robots
becoming part of our daily life's, for example in the form of 
vacuum cleaners and lawn mowers. In the future areas like care giving and
everyday assistance and household work may become successful
application domains
of robotics \citep{schaal2007new}.
This poses a dramatic shift from the prior uses of robots in industry, where
they mainly worked in isolated and predefined contexts.
Especially countries like Japan, facing the problem of an
aging population, put increasing effort to make robots
viable in these new application domains.  \\
In daily life, the robots are confronted with different challenges,
like adapting to different lighting conditions and
objects being moved around.
The robots which are currently on the market
like vacuum cleaners and lawn mowers have
either a ``one size fits all'' approach or need a special setup
in software or hardware. To solve this issue,
machine learning and especially
reinforcement learning (RL) will be key technologies to enable robots
to adjust to dynamic and stochastic environments.

Robotics and reinforcement learning complement each other
with robotics providing a real world testing ground and
reinforcement learning providing
the framework for formulating problems and finding solutions. The 
relationship may be similar to the one of  math and physics
\citep{kober2013reinforcement}. 
Compared to other domains, in which reinforcement learning has been
successfully employed,
robotics poses a unique set of challenges. This makes
it necessary to explore methods which are adjusted to the inherent
requirements of robotics, which include:

\begin{itemize}
\item a high dimensional state and action space,
\item problem of obtaining real world samples,
\item problem of goal specification,
\item and dealing with under-modeling and model uncertainty.
\end{itemize}

Robots have the potential to transform our society. By
bringing robots from the factories into our homes there will be many
possibilities for improvement. To reap the benefits of this
development we will have to overcome many technical and
ethical challenges alike.

% TODO: include pictures of robots solving tasks with RL


\section{Contribution}
Policy search algorithms have shown promise as an alternative
to value function-based reinforcement learning, especially for learning motor
skills in robotics \citep{deisenroth2013survey}.
The MORE algorithm is introduced in
\citet{abdolmaleki2015model} as a stochastic search algorithm that
can be used for policy search.
The key idea of MORE is to learn a surrogate model of
the objective function to efficiently compute the updates to
the policy in closed form. One of
the contributions of this thesis is to explore recursive estimation 
for learning the surrogate model
with the goal of increasing the sample efficiency of the MORE algorithm.
We focus on classical methods of parameter estimation and filtering
like the Recursive Least Squares algorithm and the Kalman filter,
considering more advanced methods is part of future work.
We benchmark our version of the MORE algorithm on an optimization test
function, a simple planar reaching task and on a simulation
of a ball-in-a-cup task using the barret WAM robot arm
and compare them to previous results. While we could improve the
sample efficiency on the optimization test function, we
did not reach state-of-the-art results for our robotic tasks.
Nevertheless, we could successfully learn the robotic tasks demonstrating
that the recursive surrogate-modeling approach may have the potential to
eventually yield superior results with more time and research.


\section{Structure of the Thesis}
The remainder of this thesis is structured as follows:

\textbf{Chapter 2:} In the fundamentals chapter we lay the foundation for
the thesis, first introducing the basics of reinforcement learning and
then focusing on the specifics of applying RL to robotic tasks.
Then we discuss policy search as one method for
solving the robot learning problem.
Next we introduce the original MORE algorithm and
look at Bayesian filtering. Finally we discuss some data preprocessing
techniques used.

\textbf{Chapter 3:} Review of the related work in the field.

\textbf{Chapter 4:} In this chapter, we first motivate the idea of
using recursive estimation for learning the surrogate model.
Then we present a batch solution for the
surrogate-model estimation task, which we use for comparison
in our experiments.
We present our approach of connecting
the MORE algorithm with recursive estimation techniques for the 
surrogate model.

\textbf{Chapter 5:} In the evaluation chapter we conduct experiments
with our algorithms on several test functions and some simple robot tasks.
We compare our algorithm with the original MORE algorithm
and the least squares approach for learning the surrogate model.

\textbf{Chapter 6:} We conclude the thesis with a summary of
the achieved results and an outlook on future work.