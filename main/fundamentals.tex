% !TeX spellcheck = en_US
% !TeX encoding = UTF-8
% !TeX root = ../thesis.tex

\chapter{Fundamentals}
This chapter introduces basic concepts used throughout this thesis.
First basics of reinforcement learning and the problem of robot learning.
Next we discuss policy search, a sub-field of reinforcement learning,
as one method to solve the robot learning problem.
The Kullback-Leibler divergence (KL), as an important information theoretic
distance metric between probability distributions.
Having discussed the underlying basics we can then introduce the MORE Algorithm.
Finally we will look at Filtering from a Bayesian Estimation viewpoint and
review the Kalman Filter for parameter estimation.


\section{Reinforcement Learning}
Reinforcement Learning is a subfield of Machine Learning concerned with agents
learning to interact with their environment.
This is done through exploration and trial-and-error, trying to discover
cause and effect relationship between actions.
Compared to supervised learning and unsupervised learning it
more closely resembles the way humans learn.

The term reinforcement learning relates to a class of problems, solution methods
and the field that studies these problems and solutions.
\citet{sutton2018reinforcement}

Some famous examples include playing games like Go but also atari
and also the problems like a robot learning to walk.
Generally RL is applicable to a large range of problems, making it very popular.
Whereas in supervised learning the best action is presented to the system,
the agent in a reinforcement learning setting receives an occasional
reward (or punishment).
This difference can be formulated as the concept
of learning with a critic, who gives occasional feedback, as opposed to learning
with a teacher.
To gain information about the rewards the agent needs
to explore previously unused actions. Dare to try new things or keep performing safe
well-known actions, this is the \textit{exploration-exploitation tradeoff}.
The agent should exploit actions he knows that give decent reward, but he first
has to try different things to learn about these actions, and progressively has to 
focus in on them.
The reward singal is only given occasionly, hence the amount of information
the agent receives is minimal compared to
supervised and unsupervised learning approaches.
Focus on simple general principles for learning that can be applied to different problems.

Reinforcement learning as general case of optimal control, whereas optimal control
assumes perfect knowledge, RL uses apporixmations and data-driven techniques.
We can distinguish an \textit{episodic} setting, where the task is restarted after
the end of an episode, here the goal is to maximize the total reward per episode.
Otherwise if the taks is on-going the goal is to achieve high average reward over
the whole life-time or a formulation with a sicounted return (wheighting the
future and past differently).
We model the agent and its environment as a state $s \in S$ it may perform action
$ a \in A$ which can be either discrete or continous.
For every step the agent receives a Reward $R$, which is a scalar value
The overarching goal is to find a mapping from states to actions, a policy $\pi$,
that picks actions in a way that the reward is maximized. The policy $\pi$ may
be deterministic $a = \pi(s)$ or stochastic $a \sim \pi(s,a) = P(a | s)$.

The classical approach to formalizing problems in RL is through
Markov Decision Processes (MDPs).
MDPs are a mathematical
framework for decision making in deterministic and stochastic environments.
MDPs focus on only three aspects - sensation, action and goal, which are central
for reinfocement learning problems.
MDPs satisfy the Markov property (cite), which state that ``the future is independent
of the past given the present''. In our case this means the next state $s'$ and the reward
only depend on the previous state $s$ and action $a$ \citet{sutton1992reinforcement}.
A MDP can be formally defined as a tuple $(S, A, P, r)$:

\begin{itemize}
\item a set of states $s \in S$ that describe the environment.
\item a set of actions $a \in A$ that can be performed by the agent in the environment
\item a transition function $P(s_{t+1} | s_t, a_t)$ that gives the probability of a new
  state $s_{t+1}$ after an action $a_t$ has been taken in state $s_t$
\item a reward function $r(s_t, a_t)$ that specifies the immediate reward after taking action
  $a_t$ in state $s_t$
\end{itemize}

The Markov property can be expressed as
$$ P(s_{t+1} | s_t, a_t, s_{t-1}, a_{t-1}\cdots) = P(s_{t+1} | s_t, a_t)$$
This recapitulates the notion of state - a state is a sufficient statistic
for predicting the future, rendering previous observations irrelevant.
In robotics we may only find some approximate notion of state.

Generally the goal is to find an optimal policy $\pi*$ that
maximizing the expected return $J$.
Optimal behaviour can be modeled in different ways, resulting in different
definitions for expected return.

Finite-horizon model, for horizon $H$ meaning the next $H$ timesteps:
$$ J = E \left\{\sum^H_{t=0} R_t \right\} $$

Discounted reward, discounting future rewards by a discount factor $\gamma$
(with $0 \leq gamma < 1)$:
$$ J = E \left\{\sum^{\infty}_{t=0} \gamma^t R_t \right\} $$

\section{Robot Learning}
The sub-field where reinforcement learning and machinelearning
intersect with robotics is called \textit{Robot Learning}, it aims to bridge
the gap between programmed robots,
with fine tuned controlles  and fully autnomous robots.
As proposed in \citet{deisenroth2013survey}, robot control can be modeled as
a reinforcement learning problem.

RL offers general framework, a movement is represented in terms of an optimizatoin or
reward criterion, learning algorithms fill the details of the
Robotics differes from most other RL problems, high dimensional states and actions space.

- true state not observable not noise free

- reward shaping is difficult and requires fair amount of domain knowledge

- robot chooses motor control $u$ according to policy $\pi$

- probabilistic transition function $p(x_{t+1} | x_t, u_t$

- states and actions form \textit{trajectory} $\tau = (x_0, u_0, x_1, u_1,...)$
called \textit{rollout} or a \textit{path}

- episodic reward
$$ R(\tau) = r_T(x_T) + \sum^{T-1}_{t=0} r_t(x_t,u_t) $$
with $r_t$ is instantaneous reward function (consumed energy) and
$r_T$ final reward, such as quadratic punishment term for deviation from goal posture.

- infinite-horizon case
$$ R(\tau) = \sum^{\infty}_{t=0} \gamma^t r(x_t, u_t) $$
where $\gamma \in [0,1)$ is a discount factor that discounts rewards further in the future.

- finally tasks in robot learning as chossing a optimal control policy $\pi^*$ that
maximizes the expected accumulated reward

- use imitation learning to initialize prior, this removes the need for global exploration,
  the student can improve locally

  - generally no clear receipe for robot learning, hugh number of different methods,
  but when used appropriately achive great results.

  - though minor incorrect flaws can completely prevent success of learning.

  - learning algorithms are rarely used on robots for real daily usage, and most
  algorithms are overfitted to a praticular robot architecture and do not generalize
  to other robots easily
  
\subsection{Challenges}
Robotics is different compared to other fields where Reinforcement Learning
is used extensively. We have a high dimensional state space and running
real world systems on real hardware is costly.

Nevertheless robotics provides an important testing ground for Reinforcement Learning.

\subsubsection{Curse of Dimensionality}
(continuous state and action space)

\subsubsection{Curse of Real-world Samples}
\begin{itemize}
\item training time is limited $\rightarrow$ only relatively few complete executions can ever be generated
\end{itemize}

- generating \textit{interesting} data with useful information is difficult to
obtain since we are trying to improve imperfect controller. This is also
called exploration/exploitation tradeoff

- sharing data across different robotics as \textit{cloud robotics}, for example Google's self
driving cars, but for controlling motion of robots has strong realtime requirements, and
same robot model can create different data depending on environment conditions and
wear and tear

- Ensuring safe interaction with human being (in everyday life applications),
robustness and reliability in learning

\subsubsection{Curse of Goal Specification}

\subsection{Policy search}
Policy search methods use parametrized policies $\pi_{\theta}$ and operate
directly in the parameter space $\Theta$ of theses parametrized policies and
typically avoiding learing value function.

Most traditional reinforcement methos use \textit{action-value methods}, in robotics
there has been a focus on learning \textit{parametrized policies} that can select
actions without a value function.

Policy search method use parametrized policies, instead
of value-based methods. Directly learn the policy without a value function.
This enables Imitation learning, natural integration of expert knowledge through
initialization of the policy.
This method has good results (cite reps, more, policy search review)

Still choosing the right method for the right problem remains an open question.
- Still in the robotics framework the computational cost of updating
a policy in policy sarch can easiyl be cheaper than finding just one
optimal action for one state by searching the state-action value function.

- policy search as \textit{actor}-only method and value function as \textit{critic}-only
because in value function we try to estimate performance of one control then improve
the policy on this, whereas in policy search we direcly derive optimal policy.

Most Policy search methods optimize exsiting parametrized policy locally. By computing
changes to the policy parameters that will increase the expected retunr and results
in iterative updates.

The computation of the policy is the main concept to focus on. There have
been different approaches gradient estimation, stochastic optimization,... (cite)

Generally we can divide policy search into model-free and model-based and
differentiate whether stochastic or deterministic trajectories are used.
There are model-free and model-based Policy search methods \citet{deisenroth2013survey}.

- TODO: include graphic for different policy search methods

- parametrized policy $\pi_{\theta}$, the search is performed in the parameter space
$\theta \in \Theta$

- We will focus on model free policy search methods where we sample trajectories from
the robot and improve the policy only based on the samples and the reward signal.

- model free policy search methods use sampled trajectories and immediate rewards directly
to update parameters of the policy

- mode-based methods, model the the robot's and environments's dynamics from sampled
  trajectories, and then use this model for improving the policy.

- group of algorithms stochastic search algorithms, which are \textit{black-box optimizers}.
algorithm has no knowledge of shape or gradient of the objective function.

- approach useful when objective function is unknown or too complex to model.

- applied for policy search in episode-based formulation, maintaining a upper-level policy
$\pi_{omega}(\theta)$ to create samples in the parameter space, which are
subsequently evaluated on the real system.

- simplicity and ease of use of these algorithms

- when we look at MORE algorithm we will see that stochastic optimization algorithms can
be readily be used to perform model-free policy search.
  
%%%%%%%%%%%%%%%%%%%%%%%%%%%%%%%%%%%%%%%%%%%%%%%%%%%%%%%%%%%%%%%%%%%%%%%%%%%%%%%%
%\section{Constraint Optimization}
%- look at mechano informatik slides
%
%In constraint optimization we want to maximize a function subject to constraints on
%the variables. Generally we have the following problem formulation
%
%\begin{align}
%  \min_{x\in \mathbf{R}^N} f(x)  \text{subject to}
%  \begin{cases}
%    c_i(x) = 0 \\
%    c_i(x) \geq 0
%  \end{cases}
%\end{align}
%
%
%Introducing the Lagragian function we get
%\begin{align}
% \mathcal{L}(\pi, \eta, \omega) = 
%\int \pi(\theta) \mathbf{R}_{\theta} d\theta \; + \; 
%\eta  \left(\epsilon - \int \pi(\theta) \text{ log}
% \frac{\pi(\theta)}{q(\theta)} d\theta\right)
% - \; \omega \left(\beta + \int \pi(\theta) \text{ log}(\pi(\theta)) d\theta\right)
%\end{align}
%
%We can now construct an alternative problem the dual function, which is easier to
%solve than the primal problem. We optimize for the Lagragian multipliers.
%
%%%%%%%%%%%%%%%%%%%%%%%%%%%%%%%%%%%%%%%%%%%%%%%%%%%%%%%%%%%%%%%%%%%%%%%%%%%%%%%%

\section{Kullback-Leibler (KL) Divergence}
Many algorithms in policy search rely on the Kullback-Leibler Divergence, also
kown as the relative entropy, for controlling
the difference between the the old and updated policy.
Working with real robots additionaly requires to perform safe exploration, big
exploration steps, may result in damaging the hardware.
Specifically, it measures the shannon entropy of one distribution relative to the
other.

$$ KL(p || q) = \int p(\theta) \text{log} \frac{p(\theta)}{q(\theta)} d \theta $$

Note that in general the relative entropy is not symmetric under interchange of the
distributions $p$ and $q$. In general $KL(p || q) \neq KL(q || p) $, therefore
in a mathematically sense it is not stricly a distance.

The KL-bound has been used successfully for many algorithms (cite REPS, Natural gradient)
It is widely used in Machine Learning algorithms. (cite..)

\section{MORE Algorithm}
The MORE algorithm is a stochastic search algorithm. It can be used for
model-free policy search. The key idea is using information-theoretic policy updates
by bounding the relative entropy (Kullback Leibler divergence) between two subsequent
policies. We approximate the objective function with a quadratic surrogate model hence
we can statisfy the KL-bound in closed form.


- TODO: include 3D figure of more algorithm

\subsection{MORE Framework}
The MORE algorithm maintains a search distribution (policy) over the parameterspace
of the objective function. We can use the method of constraint optimization
to formulate an optimization problem for obtaining a new search distribution that
maximizes the expected objective value while upper-bounding the KL-divergence and
lower-bounding the entropy of the distrubtion.

\begin{align}
 \max_{\pi} \int \pi(\theta) \mathbf{R}_{\theta} d\theta, \\
\; \text{ s.t. KL}(\pi(\theta)||q(\theta)) \leq \epsilon, \\
\quad H(\pi) \geq \beta, \\
\quad 1 = \int \pi(\theta) d\theta
\end{align}

Forming the Lagragian we get
\begin{align} \mathcal{L}(\pi, \eta, \omega) = 
\int \pi(\theta) \mathcal{R}_{\theta} d\theta \; + \; 
\eta  \left(\epsilon - \int \pi(\theta) \text{ log}
 \frac{\pi(\theta)}{q(\theta)} d\theta\right)
 - \; \omega \left(\beta + \int \pi(\theta) \text{ log}(\pi(\theta)) d\theta\right)
\end{align}

The solution is
\begin{align}
\pi(\theta) \propto q(\theta)^{\eta/(\eta+\omega)} 
\text{exp}\left(\frac{\mathcal{R}_\theta}{\eta + \omega}\right)
\end{align}
it depends on the quadratic and linear term of the surrogate model and the
lagragian mutlipliers. These can be obtained by minimizing the dual problem:

\begin{align}
  g(\eta,\omega) = \eta\epsilon - \omega\beta + (\eta - \omega) \text{log}
\left(\int q(\theta)^{\frac{\eta}{\eta + \omega}}
  \text{exp}\left(\frac{\mathcal{R}_\theta}{\eta + \omega}\right) d\theta \right)
\end{align}
We get the closed form solution for the equation
(see full derivation in Appendix).

With these equations we can formulate an iterative algorithm for updating the policy.

\subsubsection{Entropy constraint}
The bound $\beta$ is defined such that realtive difference between the entropy
of the policy $H(\pi)$ and a minimum exploration policy $H(\pi_0)$ is decreased
for a certain percentage:
$$ H(\pi) - H(\pi_0) \geq \gamma (H(q) - H(\pi_0))
\rightarrow \beta = \gamma (H(q) - H(\pi_0) + H(\pi_0) $$

\subsection{Surrogate Model}
The key idea of the MORE algorithm is using a surrogate model of the objective function
for satisfying the bound set by the optimization problem. It has proven to be
sufficient to use a quadratic model, since the exponent of Gaussian distribution
is also quadratic, it is not possible to exploit the information of a more
complex surrogate model. This concept of using a surrogate model has been used
in numerical optimization (cite form book numerical optimization)

The original approach of MORE is based on weighted Bayesian linear model with
dimensionality reduction.

\subsection{Analytic Solution for Dual-Function and Policy}
Assuming we are given a quadratic surrogate model
$$ R_\theta \approx \theta^T R \theta + \theta^T r + r_0 $$
we can now solve the dual function in closed form.

$$ g(\eta, \omega) = \eta \epsilon - \beta \omega
+ \frac{1}{2} \left(\mathbf{f}^T \mathbf{F} \mathbf{f} - \eta \mathbf{b}^T \mathbf{Q}^{-1}
\mathbf{b} - \eta log |2\pi \mathbf{Q}| + (\eta + \omega) log |2\pi (\eta + \omega)
\mathbf{F}| \right) $$

Assuming $\pi$ is gaussian we get the solution, and can update the mean and covariance matrix
of the search distribution. 
\begin{align} \label{policy_update}
  \pi_{t+1} &= \mathcal{N}(\mu_{t+1}, \Sigma_{t+1}) \\
  \mu_{t+1} &= (\eta \Sigma_{t}^{-1}\mu_t + r) / (\eta + \omega) \\
  \Sigma_{t+1} &= (\eta \Sigma_t^{-1} + R) / (\eta + \omega)
\end{align}

With these equations we can iteratively update our search distribution.
It is important to note that MORE can be used as a stochastic search algorithm,
a ``black box solver'' because only the samples and objective function value
is needed to update the search distribution.
In Robotics it can be used to learn motor tasks by using Dynamic Movement Primitives
(DMPs) as a policy representation.

\section{Bayesian Filtering}

TODO:  rewrite, because currently this part to closely resembles \citet{sarkka2013bayesian}.

Optimal filtering is concerned with estimating the state of a time-varying system
which is only partialy observed through noisy meaurements.
\textit{Bayesian Filtering} refers to the Bayesian way of formulating optimal
filtering.

In general the term ``Bayesian'' refers to inference methods that represent
``degrees of certainty'' using probability theory, and basically use \textbf{Bayes' rule}
to update the degree of certainty given data.

Filtering methods are important in robotics for handling the uncertainty in the
environment. Sensing and modeling the state of the robot itself.
Methods like the Kalman Filter are extensively used since we are dealing with noisy sensors.

Also the data is very limited and when the robots needs to make
decisions it can only work with small amounts of data, therefore
therefore Bayesian methods are a great fit and are widely used (cite..)

In general we solve linear Gaussian and non-linear/non-Gaussian state space
models with the Bayesian filtering equations.

- optimal recursive estimators first presented for linear systems (simplicity)

- most natural optimality criterion is the least squares optimality,
minimum mean squared error

- statistical inversion problem: estimate hidden states from observed
measurements in Bayesian sense we want to compute joint posterior distribution.
Straightforward application of Bayes rule

- Bayesian viewpoint sees probabilities as degree of belief.

- Bayes rule can be used to update our information about a quantity

- Bayesian probability building blocks

\subsection{Building Blocks of Bayesian models}
In general all Bayesian models have some basic components to them.

\subsubsection{prior distribution}
Encodes the information on parameter $\theta$ before seeing any
observations. When we are uncertain about our prior information
we can choose a high variance of the distribution or use a
non-informative prior (which imposes the minimal amount of structure
on the data).
$$ p(\theta) = \text{information on parameter } \theta
\text{before seeing any observations} $$

\subsubsection{measurement model}
Models the relationship between true parameters and the measurements.
$$ p(y | \theta) = \text{distribution of observation } y
\text{given the parameters} \theta $$

\subsubsection{posterior distribution}
The conditional distribution of the parameters given the observations.
It represents the updated belief about the parameters
after obtaining the measurments. It can be computed by using Bayes' rule.

$$ p(\theta | y) = \frac{p(y | \theta) p(\theta)}{p(y)}
\propto p(y | \theta) p(\theta) $$

The normalization constant can be computed by integrating $\theta$ out.

$$ p(y) = \int p(y | \theta) p(\theta) d\theta $$

-TODO: create graphic (2d or 3d) for merging prior with observation to get
    posterior (show with multivariate gaussian distribution)

\subsection{Gaussian probability function}
A random variable $x \in \mathbb{R}^n$ has a Gaussian distribution with mean
$m \in \mathbb{R}^n$ and covariance $P \in \mathbb{R}^{n\times n}$ if its
probability density has the form
$$ \text{N}(x | m, P) = \frac{1}{(2\pi)^{n / 2} |P|^{1/2}}
\text{exp} \left( -\frac{1}{2} (x - m)^T P^{-1} (x-m) \right) $$
where $|P|$ is the determinant of the marix $P$.
    
\subsection{Least squares}
To grasp the Bayesian Filtering viewpoint we will now look
at a simple regression problem from a Bayesian perspective
and derive the least squares solution.

We consider a simple linear regression problem,
$$ y_k = \theta_1 + \theta_2 x_k + \epsilon_k $$

note that the response $y_k$ is only dependent on one regressor $x_k$.

Writing this in probabilistic terms we get:
\begin{align}
  p(y_k | \theta) &= \text{N}(y_k | H_k \theta, \sigma^2) \\
  p(\theta) &= \text{N}(\theta | \text{m}_0, \text{P}_0)
\end{align}

Here $H_k = (1 x_k)$ and N$(\dot)$ is the Gaussian probability density function.
The batch solution can then be easily obtained by application of Bayes'rule

$$ p(\theta | y_{1:T})  \propto p(\theta) \prod^T_{k=1} p (y_k | \theta) $$
$$ = N(\theta | m_0, P_0) \prod^T_{k=1} N(y_k | \textbf{H}_k \theta, \sigma^2) $$

Because the prior and likelihood are Gaussian, the \textit{posterior distribution}
will also be Gaussian
$$ p(\theta | y_{1:T}) = \text{N}(\theta | m_t, P_t) $$

Mean and covariance obtained by completing the quadratic form in the exponent:
- TODO: Do calculation in Appendix, (use canonical form)

\begin{align}
  \mathbf{m}_T &= \left[ \mathbf{P}^{-1}_0 + \frac{1}{\sigma^2} \mathbf{H}^T \mathbf{H}
                 \right]^{-1} \left[\frac{1}{\sigma^2} \mathbf{H}^T \mathbf{y} +
  \mathbf{P}^{-1}_0 \mathbf{m}_0 \right] \\
  \mathbf{P}_T &= \left[\mathbf{P}_0^{-1} + \frac{1}{\sigma^2} \mathbf{H}^T \mathbf{H}
                 \right]^{-1}
\end{align}

This gives us the batch least squares solution.

- TODO talk more about how to why this is ordinary least squares?

\subsection{Recursive least squares} \label{RLS}
With the Bayesian formulation we can now  introduce a recursive
solution to the regression problem, resulting in a recursive least squares
algorithm.

For this we have to assume the next step only
depends on the previous state, (markovian assumption).
The main idea is using the previous posterior as the prior in the estimation
step.
The measurements are assumed to be conditionally independnt.

To get the \textit{recursive Bayesian solution} we proceed as follows:
In the beginning of estimation all the information about the parameter $\theta$
are contained in the prior distribution $p(\theta)$.
The measurements are obtained on at a time, to
obtain the next posterior distribution we use the measurement and the information
from the previous posterior distribution as the prior.

\begin{align*}
  p(\theta | y_{1}) &= \frac{1}{Z_1} p(y_1 | \theta) p(\theta) \\
  p(\theta | y_{1:2}) &= \frac{1}{Z_2} p(y_2 | \theta) p(\theta | y_1) \\
                    &\vdots \\
  p(\theta | y_{1:T}) &= \frac{1}{Z_T} p(y_T | \theta) p(\theta | y_{1:T-1})
\end{align*}
Normalization term: $Z_k = \int p(\theta) p(y_k | \theta) d\theta$ % normalize likelihood 


Using this approach to solve our simple linear regression model we get
\begin{align}
  p(\theta | y_{1:k}) &\propto p(y_k | \theta) p(\theta | y_{1:k-1}) \\
                      &\propto \text{N}(\theta | \mathbf{m}_k, \mathbf{P}_k)
\end{align}
from which we get:
\begin{align}
  \mathbf{m}_k &= \left[ \mathbf{P}^{-1}_{k-1}
                 + \frac{1}{\sigma^2} \mathbf{H}^T_k \mathbf{H}_k \right]^{-1}
                 \left[\frac{1}{\sigma^2} \mathbf{H}^T_k \mathbf{y}_k +
                 \mathbf{P}^{-1}_{k-1} \mathbf{m}_{k-1} \right] \\
  \mathbf{P}_k &= \left[\mathbf{P}_{k-1}^{-1}
                 + \frac{1}{\sigma^2} \mathbf{H}^T_k \mathbf{H}_k \right]^{-1}
\end{align}

TODO: do calculation with matrix inversion lemma (in appendix?)

By using the \textit{matrix inversion lemma} the covariance caculation can be
written as
$$ P_k = P_{k-1} - P_{k-1} H_k^T [H_k P_{k-1} H_k^T + \sigma^2]^{-1} H_k P_{k-1} $$

By introducing temporary varaiables $S_k$ and $\mathbf{K}_k$ the caclulation of
the mean and covariance can be written in the form
\begin{align*}
     S_k &= \textbf{H}_k \textbf{P}_{k-1} \textbf{H}^T_k + \sigma^2 \\
     \textbf{K}_k &= \textbf{P}_{k-1} \textbf{H}^T_k S_k^{-1} \\
     \textbf{m}_k &= \textbf{m}_{k-1} + \textbf{K}_k [y_k - \textbf{H}_k \textbf{m}_{k-1}] \\
     \textbf{P}_k &= \textbf{P}_{k-1} - \textbf{K}_k S_k \textbf{K}_k^T
\end{align*}
Here $\mathbf{K}$ is the Kalman gain that controls the weight we give to new
measurments.
This is actually a special case of The Kalman filter update equations.
Since here the parameters $\theta$ are assumed to stay constant there is
no stochastic dynamics model used for the prediction step.

For estimating the surrogate model for the MORE algorithm we have to
deal with dynamic parameters instead. In the next chapter we will therefore
introduce a drift model for the parameters.

\subsubsection{Drift model}
We assume the parameter $\theta$ performs a \textit{Gaussian Random Walk} after
each time step.
\begin{align}
  p(y_k | \theta) &= \text{N}(y_k | \mathbf{H}_k \theta_k, \sigma^2) \\
  p(\theta_k | \theta_{k-1}) &= \text{N} (\theta_k | \theta_{k-1}, Q) \\
  p(\theta_0) &= \text{N}(\theta_0 | \mathbf{m}_0, \mathbf{P}_0)
\end{align}

Where $\mathbf{Q}$ is the covariance of the random walk. 
Starting with the distribution

$$ p(\theta_{k-1} | y_{1:k-1})
= \text{N}(\theta_{k-1} | \mathbf{m}_{k-1}, \mathbf{P}_{k-1}) $$

the joint distribution of $\theta_k$ and $\theta_{k-1}$ is

$$ p(\theta_k, \theta_{k-1} | y_{1:k-1})
= p(\theta_k | \theta{k-1}) | y_{1:k-1})
$$

The distribution of $\theta_k$ given the measurement history up to time
step $k - 1$ can be calculated by integrating over
$\theta_{k-1}$:
$$ p(\theta_k | y_{1:k-1})
= \int p(\theta_k | \theta_{k-1}) p(\theta_{k-1} | y_{1:k-1}) d\theta_{k-1}
$$

This relationship is sometimes called \textit{Chapman-Kolmogorov equation}
The result of the marginalization is Gaussian

$$ p(\theta_k | y_{1:k-1}) = \text{N}(\theta_k | m_k^{-}, P_k^{-}) $$

with
\begin{align}
  m_k^{-} &= m_{k-1} \\
  P_k^{-} &= P_{k-1} + Q
\end{align}

Now we simply replace $P_{k-1}$ with $P_k^{-}$ in our update equations:
\begin{align*}
     S_k &= \textbf{H}_k \textbf{P}_{k}^{-} \textbf{H}^T_k + \sigma^2 \\
     \textbf{K}_k &= \textbf{P}_{k}^{-} \textbf{H}^T_k S_k^{-1} \\
     \textbf{m}_k &= \textbf{m}_{k}^{-} + \textbf{K}_k [y_k - \textbf{H}_k \textbf{m}_{k}^{-}] \\
     \textbf{P}_k &= \textbf{P}_{k}^{-} - \textbf{K}_k S_k \textbf{K}_k^T
\end{align*}


These equations are a special case of the Kalman Filter, only doing
a adding noise to the parameters in the prediction step and using
an identity matrix as the state transition model. There is no
obvious state transition model for the general case, maybe some
method based on an momentum approach on the differences in subsequent
estimated parameters. Otherwise this could be engineered to a certain
objective function, but this is part of future work.